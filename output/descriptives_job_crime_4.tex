% latex table generated in R 3.6.0 by xtable 1.8-4 package
% Sat Apr 25 05:28:42 2020
\begin{table}[htp]
\footnotesize
\setlength{\tabcolsep}{10pt}
\renewcommand{\arraystretch}{1.3}
\begin{threeparttable}
\centering
\caption{Socio-demographic characteristics of women inmates \newline by four crime-employment clusters (N =207)} 
\label{tab:descriptives_job_crime_4}
\begin{tabular}{lcccc}
  \hline
Variable & Unemployed & Offenders & Self-employed & Employed \\ 
  \hline
Age* & 36.27 & 31.57 & 43.26 & 37.24 \\ 
  High school & 0.28 & 0.11 & 0.37 & 0.51 \\ 
  Number of children* & 2.49 & 2.32 & 3.12 & 2.32 \\ 
  Worked before prison & 0.43 & 0.19 & 0.72 & 0.78 \\ 
  Number of previous sentences* & 2.58 & 9.17 & 2.67 & 1.08 \\ 
  Sentence length in months* & 2.48 & 0.86 & 2.38 & 3.35 \\ 
  Dependence / abuse of drugs & 0.41 & 0.68 & 0.14 & 0.24 \\ 
  Mental health problems* & 0.06 & 0.11 & 0.05 & -0.18 \\ 
  Searched for jobs follow-up & 0.41 & 0.15 & 0.34 & 0.65 \\ 
  Prison during follow-up & 0.28 & 0.55 & 0.02 & 0.05 \\ 
   \hline
\end{tabular}
\begin{tablenotes}
\scriptsize
\item All the values are proportions except for * that are averages.
\end{tablenotes}
\end{threeparttable}
\end{table}
