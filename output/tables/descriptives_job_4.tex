% latex table generated in R 3.6.0 by xtable 1.8-4 package
% Sun May  3 07:30:40 2020
\begin{table}[htp]
\footnotesize
\setlength{\tabcolsep}{10pt}
\renewcommand{\arraystretch}{1.3}
\begin{threeparttable}
\centering
\caption{Socio-demographic characteristics of women inmates \newline by four employment clusters (N =207)} 
\label{tab:descriptives_job_4}
\begin{tabular}{lcccc}
  \hline
Variable & Unemployed & Self-employed & Under-the-table & Legitimate employed \\ 
  \hline
Age* & 34.21 & 42.11 & 41.68 & 32.94 \\ 
  High school & 0.23 & 0.34 & 0.32 & 0.65 \\ 
  Number of children* & 2.32 & 3.15 & 2.77 & 2.18 \\ 
  Worked before prison & 0.32 & 0.74 & 0.64 & 0.82 \\ 
  Number of previous sentences* & 5.26 & 3.20 & 1.32 & 0.94 \\ 
  Sentence length in months* & 1.87 & 2.05 & 3.22 & 3.65 \\ 
  Dependence / abuse of drugs & 0.52 & 0.17 & 0.27 & 0.24 \\ 
  Mental health problems* & 0.12 & -0.02 & -0.11 & -0.27 \\ 
  Searched for jobs follow-up & 0.45 & 0.40 & 0.59 & 0.88 \\ 
  Prison during follow-up & 0.38 & 0.06 & 0.05 & 0.18 \\ 
   \hline
\end{tabular}
\begin{tablenotes}
\scriptsize
\item All the values are proportions except for * that are averages.
\end{tablenotes}
\end{threeparttable}
\end{table}
