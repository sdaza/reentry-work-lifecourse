% latex table generated in R 3.6.0 by xtable 1.8-4 package
% Mon Dec 30 17:08:43 2019
\begin{table}[htp]
\footnotesize
\setlength{\tabcolsep}{10pt}
\renewcommand{\arraystretch}{1.3}
\begin{threeparttable}
\centering
\caption{Descriptive statistics by employment-crime clusters \newline based on solution in Figure \ref{fig:sequences_job_crime_clusters_4_v2} (N = 207)} 
\label{tab:descriptive_job_crime_clusters_4}
\begin{tabular}{lcccc}
  \hline
Variable & Cluster 1 & Cluster 2 & Cluster 3 & Cluster 4 \\ 
  \hline
Age* & 31.54 & 35.87 & 38.46 & 43.00 \\ 
  High school & 0.18 & 0.24 & 0.54 & 0.36 \\ 
  Number of children* & 2.33 & 2.46 & 2.49 & 3.00 \\ 
  Worked before prison & 0.18 & 0.43 & 0.78 & 0.76 \\ 
  Number of previous sentences* & 9.04 & 2.60 & 1.11 & 2.18 \\ 
  Dependence / abuse of drugs & 0.72 & 0.38 & 0.22 & 0.13 \\ 
  Mental health problems* & 0.13 & 0.14 & -0.17 & -0.11 \\ 
  Searched for jobs follow-up & 0.21 & 0.62 & 0.73 & 0.42 \\ 
  Prison during follow-up & 0.51 & 0.25 & 0.05 & 0.02 \\ 
   \hline
\end{tabular}
\begin{tablenotes}
\scriptsize
\item All the values are proportions except for * that are averages.
\end{tablenotes}
\end{threeparttable}
\end{table}
