% latex table generated in R 3.6.0 by xtable 1.8-4 package
% Mon Dec 30 17:08:43 2019
\begin{table}[htp]
\footnotesize
\setlength{\tabcolsep}{10pt}
\renewcommand{\arraystretch}{1.3}
\begin{threeparttable}
\centering
\caption{Descriptive statistics by employment clusters \newline based on solution in Figure \ref{fig:sequences_job_clusters_4} (N = 207)} 
\label{tab:descriptive_job_clusters_4}
\begin{tabular}{lcccc}
  \hline
Variable & Cluster 1 & Cluster 2 & Cluster 3 & Cluster 4 \\ 
  \hline
Age* & 34.02 & 33.94 & 42.10 & 41.50 \\ 
  High school & 0.22 & 0.71 & 0.34 & 0.35 \\ 
  Number of children* & 2.38 & 2.06 & 3.06 & 2.70 \\ 
  Worked before prison & 0.30 & 0.82 & 0.76 & 0.70 \\ 
  Number of previous sentences* & 5.28 & 1.00 & 3.12 & 1.30 \\ 
  Dependence / abuse of drugs & 0.53 & 0.18 & 0.18 & 0.25 \\ 
  Mental health problems* & 0.15 & -0.29 & -0.10 & -0.09 \\ 
  Searched for jobs follow-up & 0.42 & 0.88 & 0.42 & 0.65 \\ 
  Prison during follow-up & 0.36 & 0.12 & 0.06 & 0.05 \\ 
   \hline
\end{tabular}
\begin{tablenotes}
\scriptsize
\item All the values are proportions except for * that are averages.
\end{tablenotes}
\end{threeparttable}
\end{table}
